% Класс документа пока не окончательный, сильно сомневаюсь, что article 		
\documentclass[a4paper,12pt]{extarticle} 
% Подключаем шрифты,кодировки,русские переносы
\usepackage{cmap}
% подключается пакет, позволяющий улучшить вид пдф документа(как я понял)
\usepackage[utf8x]{inputenc}
% подключаем кодировку шрифтов для вносимых файлов
\usepackage[T2A]{fontenc}
% подключаем кодировку внутреннних шрифтов
\usepackage[main=russian,english]{babel}
% подключаем перенос и распознование слов, русский в приоритете
\usepackage{indentfirst}
% Отступ в начале абзаца
% \usepackage[unicode=true,pdfusetitle,
 	% bookmarks=true,bookmarksnumbered=false,bookmarksopen=true,
 	% breaklinks=false,pdfborder={0 0 0},pdfborderstyle={},backref=slide,colorlinks=false]
 	% {hyperref}
 % Надо разобраться с опциями, копирнул из пред.файла, который еще был написан в lyx
\usepackage{subfig}
% попытка загрузить minipage
\usepackage{
	amssymb,
	amsfonts,
	amsmath,
}
\usepackage{nicefrac}
% Пакеты американского математ. сообщества, красивый вид формул и текста внутри, а также дробный вид формул
\usepackage{esdiff}
% Пакет для производных
\usepackage{
	wrapfig,
	graphicx,
	caption,
	% subcaption,
	tikz,
}
\captionsetup{format=plain,labelsep=period}
% Обтекаемые объекты, рисунки, подписи без двоеточий и прочее
\usepackage{
	pgfplotstable,
	pgfplots,
	booktabs,
	colortbl,
	array,
	% float
}
\pgfplotsset{compat=newest}
% таблицы, графики
\usepackage[final]{pdfpages}
% Для вставки pdf файлов
\usepackage{geometry}
\usepackage{fancyhdr}
% границы, контитулы, и прочее


\geometry
	{
	left=1.5cm,
	right=1.5cm,
	bottom=2cm,
	top=2cm,
	}
% границы документа

\usepackage{setspace}
% убирает гигантские размеры оглавления
\linespread{1.3}
% междустрочный интервал

\pagestyle{fancy}
\fancyhead{}
% пустая шапка контитула
\fancyhead[R]{\authors}
% На правой стороне страницы авторы и науч.рук.
\fancyhead[L]{\labname}
 % Слева название лабы
\fancyfoot{}
\fancyfoot[C]{\thepage}
% номер страницы снизу по середине
\renewcommand{\contentsname}{Оглавление}
% переводим на русский язык оглавление
\usepackage{secdot}
\sectiondot{subsection}
% Ставит злосчастные точки в главах, ибо не по госту
% Преамбула почти слизана у Федора Сарафанова https://github.com/FedorSarafanov/RLC/blob/master/text/diss.tex
